%!TEX program = xelatex
%!TEX spellcheck = en_GB
\documentclass[final]{article}
\input{../../.library/preamble.tex}
\input{../../.library/style.tex}
\addbibresource{../../.library/bibliography.bib}
\begin{document}
\section{Implementation}
\label{sec:implmentation}
% Implementation (how you did it, and who did what)
Implementation (how you did it, and who did what)

the size of the C code

the individual contributions of each team member (no specification = no contribution)

%TODO[E]: Include the latencies here?
%TODO[E]: Is this also the place for the sensor noise?

\subsection{Communication}
%TODO[E]: For the packets, receive(), handlepacket() WIP
The communication is based on \SI{20}{\byte} fixed length packets.
This size was chosen to properly fit into Bluetooth Low Energy frames.
The packet layout is as shown in \cref{}. %TODO[E]: Packet layout tikz image.
The provided receive and transmit queues were used for the implementation, with a altered depth of \num{240} to fit an whole amount of \SI{20}{\byte} packets.
The receive function scans the input with a constant moving window for the header bytes, and when found receives the packet and passes it to handle packet.
If the type of packet requires an acknowledgement, the \texttt{ackNumber} field is copied and an acknowledgement packet is sent.


\subsection{System}
%TODO[E]: For the "glue" on the ES side (tick, busywork, the main loop, set_mode, timers etc.) WIP
The system is built using a continuous loop (\texttt{void busywork()}) that runs as fast as possible.
And a timer that set a flag at \SI{100}{\hertz}, this triggers the \texttt{void tick()} function.
A hardware interrupt triggers the retrieval of the DMP sensor values from the MPU, also at \SI{100}{\hertz}.
This results in a near perfect \SI{100}{\hertz} execution rate of the tick function with a maximum delay of one DMP retrieval latency, as shown in \cref{}. %TODO[E]: Add latency table

The main function that handles mode changes and check whenever they are valid is \texttt{void set\_mode(FlightMode)} where \texttt{FlightMode} is the type of the flight mode parameter.
This function returns depending on the current mode the state of the inputs and/or calibration values whenever or not the mode switch is acceptable.
If it's not, an exception packet is sent.

\subsection{Control}
%TODO[E]: The functions control (and possibily control_fast)

\subsection{Logging}
%TODO[E]: The flash dump and flash writing stuff and the PC side parsing (now in Python using struct package)
All important information is written on a per-event basis, so telemetry data is written when telemetry is sent, sensor data is written when it is retrieved and so on.
The packets are variable length and do not contain a checksum.
The first thing written is a time stamp and then the packet type, this makes sure that parsing the binary back will be possible and straightforward.
When the flash is full the logging automatically disengages, and became the function state is included in the telemetry packet to the PC the PC also knows this.
When the dumping mode is activated the transmit buffer is filler to half, and the packets are put in the buffer as quickly as the transmission allows.
This makes for a quick transmission of the full binary data.

\subsection{Dashboard}
%TODO[E]: For the PC side of things.
The Dashboard application is written in C\# for ease and speed of development, working with multiple thread is decidedly easier and more straightforward.
The main units are as follows.
%TODO[E]: Diagram?
\begin{itemize}
 \item UI/Main thread
 \begin{itemize}
    \item Configuration management
    \item Global Data reference management
    \item Controller
 \end{itemize}
 \item Input manager thread
 \begin{itemize}
    \item Polling of all acquired joystick and keyboard devices
    \item Generalized Input Event generation
    \item PatchBox to match generalised input event to application action/axis
 \end{itemize}
 \item Serial RX thread
 \item Serial TX thread
 \item Serial Event thread
 \begin{itemize}
    \item Generating generalized serial events
    \item Handling and processing incoming packets
    \item Serializing and sending outgoing packets
 \end{itemize}
 \item RC Timer thread
 \begin{itemize}
    \item Generating periodic RC value sending events
 \end{itemize}
 \item HTTP server thread
 \begin{itemize}
    \item Handling client connection for phone control
    \item Handling incoming API requests with new control information from a phone
 \end{itemize}
 \item HTTP timeout timer event thread
 \begin{itemize}
    \item Monitor client request intervals to disassociate the client and return control to the joystick.
 \end{itemize}
 \item Framework and UI binding helper threads (many)
 \begin{itemize}
    \item Process UI bindings and animation
    \item Helper thread for internal framework processes
    \item Debug threads
 \end{itemize}
\end{itemize}
\section{Work division}
The authoring tags are written in the header files where possible.
But due to the limited nuance allowed in said tags, a quick overview is given in \cref{tab:work-division}.

\begin{table}[H]
    \caption{Work division}
    \label{tab:work-division}
    \centering

    \begin{tabular}{lp{10cm}}
    \toprule
    Module                   & Notes \\
    \midrule
    ES-side architecture & Provided or tweaked/written by Robin \\
    ES-side control & Robin wrote most of the control code for full control, with Casper taking the mixing of the motors. \\
    ES-side communication & Written in part by Erwin and Robin. Initial translation of the C\# classes done by Robin.\\
    ES-side state handling & Initial design by Casper, written by Robin \\
    ES-side logging & Written by Casper \\
    ES-side Log dumping & Initial code by Casper, final finishing touches by Erwin \\
    Other ES-side code & Either provided or tweaked/written by Robin and tweaks by Erwin \\
    \midrule
    PC-side architecture & Written by Erwin \\
    PC-side control & Written by Erwin \\
    PC-side communication & Written by Erwin \\
    PC-side graphical user interface & Written by Erwin, Initial binding design and ViewModel by Robin \\
    PC-side visualization & Written by Erwin \\
    PC-side log dumping & Written by Erwin \\
    PC-side http server & Written by Erwin \\
    Other PC-side modules & Written by Erwin, tweaks by Robin \\
    \midrule
    Phone-side input & Written by Robin, tweaks by Erwin \\
    SDL UI prototype & Written by Robin \\
    Protocol prototypes & Written by Erwin \\
    Flash binary parsing & Written by Erwin \\
    \bottomrule
    \end{tabular}
\end{table}
\end{document}