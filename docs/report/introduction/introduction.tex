%!TEX program = xelatex
%!TEX spellcheck = en_GB
\documentclass[final]{article}
\input{../../.library/preamble.tex}
\input{../../.library/style.tex}
\addbibresource{../../.library/bibliography.bib}
\begin{document}
\section{Introduction}
% Introduction (including problem statement)

%Manual:
% The goal of the 6 ECTS credits course is not to have the student master the multidisciplinary skills of embedded systems engineering, but rather to have the student understand the basic principles and problems, develop a systems view, and to become reasonably comfortable with the complementary disciplines.
The goal of the Embedded Real Time Systems course is to get familiar with skills and tasks that are often involved in designing an embedded system.
So whereas one would usually just focus on the Embedded Software part of a project, this course also want you to look at the complementary disciplines.
In order to do so, the main part of this course consists of a lab where groups of 3 students have to develop the embedded software for a quad-rotor drone.
For this lab each group is given a quad-rotor drone and a joystick, the rest is up to the group to make.
So for example this includes a PC-interface which communicates with the drone using a self-designed protocol. This drone uses the joystick input to control its position.
Using some control system theory, a control system has to be designed to automatically stabilize the drone.
\end{document}