%!TEX program = xelatex
%!TEX spellcheck = en_GB
\documentclass[final]{article}
% Include all project wide packages here.
%\usepackage{fullpage}
\usepackage[a4paper,margin=2.5cm,top=2cm]{geometry}
\usepackage{polyglossia}
\setmainlanguage{english}
\usepackage{csquotes}
\usepackage{graphicx}
\usepackage{pdfpages}
\usepackage{caption}
\usepackage[list=true]{subcaption}
\usepackage{float}
\usepackage{standalone}
\usepackage{import}
\usepackage{tocloft}
\usepackage{wrapfig}
\usepackage{authblk}
\usepackage{array}
\usepackage{booktabs}
\usepackage[title,titletoc]{appendix}
\usepackage{fontspec}
\usepackage{pgfplots}
\usepackage{tikz}
\usepackage[binary-units=true,quotient-mode=fraction]{siunitx}
\usepackage{units}
\usepackage{amsmath}
\usepackage{mathtools}
\usepackage{unicode-math}
\usepackage{rotating}
\usepackage[compact]{titlesec}
\usepackage{titletoc}
\usepackage{blindtext}
\usepackage{color}
\usepackage{enumitem}
\usepackage{tabularx}
\usepackage{titling}
\usepackage[final]{microtype}
\usepackage[%
siunitx,
fulldiodes,
europeanvoltages,
europeancurrents,
europeanresistors,
americaninductors,
smartlabels]{circuitikz}

\newcommand{\matlab}{{\textsc{matlab }}}

\usetikzlibrary{calc}
\usetikzlibrary{positioning}
\usetikzlibrary{automata}
\usetikzlibrary{arrows.meta}

\tikzstyle{every state}=[fill=tu-cyan,align=center,draw=black,line width=1pt,node distance=3cm,minimum width = 1.8cm]%for FSMs casper
\tikzstyle{every initial by arrow}=[initial text={Reset}]
\newcommand{\setpathasarrows}{\tikzstyle{every path}=[auto,line width=1.5pt,line cap=round,line join=round]}

\pgfplotsset{compat=newest}
\pgfplotsset{plot coordinates/math parser=false}
\usetikzlibrary{plotmarks}
\usepgfplotslibrary{patchplots}
\newlength\figureheight
\newlength\figurewidth
\newlength\bytewidth
\newlength\byteheight
\setlength\bytewidth{1cm}
\setlength\byteheight{6ex}

\tikzset{every axis/.style={xticklabel style={align=right}}}

\tikzstyle{byte}=[draw, fill=blue!20, minimum width=6cm,line width=0.4pt,node distance=-0.4pt, minimum height=\byteheight,text width=6cm,font={\fontsize{9pt}{0}\selectfont}]
\tikzstyle{data}=[byte, fill=red!20]
\tikzstyle{padding}=[byte, fill=green!20]
\tikzstyle{byte_small}=[draw, fill=blue!20, minimum width=1.5cm,line width=0.4pt,node distance=-0.4pt, minimum height=\byteheight,text width=1.5cm]
\tikzstyle{data_small}=[byte_small, fill=red!20]

\usepackage[
%backend=bibtex,
backend=biber,
	texencoding=utf8,
bibencoding=utf8,
style=numeric,
citestyle=numeric,
    sortlocale=en_US,
    language=auto,
    backref=true,
    abbreviate=false,
    seconds=true,
    date=edtf
]{biblatex}


\usepackage{listings}
\newcommand{\includecode}[4][c]{\lstinputlisting[caption=#2, escapechar=, style=#1,label=#4]{#3}}
\newcommand{\superscript}[1]{\ensuremath{^{\textrm{#1}}}}
\newcommand{\subscript}[1]{\ensuremath{_{\textrm{#1}}}}


\newcommand{\chapternumber}{\thechapter}
\renewcommand{\appendixname}{Appendix}
\renewcommand{\appendixtocname}{Appendices}
\renewcommand{\appendixpagename}{Appendices}


\setlist[enumerate]{labelsep=*, leftmargin=1.5pc}
\setlist[enumerate,1]{label=\arabic*., ref=\arabic*}
\setlist[enumerate,2]{label=\arabic*.,ref=\theenumi.\arabic*}
\setlist[enumerate,3]{label=\arabic*., ref=\theenumii.\arabic*}

%\setcounter{chapter}{-1} %start chapter numbers with 0

\usepackage{xr-hyper}
\usepackage[hidelinks]{hyperref} %<--------ALTIJD ALS LAATSTE
\usepackage[nameinlink,noabbrev,capitalise]{cleveref} %<------- Clever Ref moet na hyperref
\crefname{app}{Appendix}{Appendices}
%\renewcommand{\familydefault}{\sfdefault}


\setmainfont{Myriad Pro}[Ligatures={Common,TeX}]
%\setmathfont{Asana Math}
\setmathfont{Asana-Math.otf}
\setmonofont[Scale=0.9]{Lucida Console}
\newfontfamily\headingfont{Minion Pro}[Ligatures={Common,TeX}]


\sisetup{detect-all}

%Design colors
\definecolor{accent1}{RGB}{0,100,200}
\definecolor{accent2}{RGB}{0,50,100}
\definecolor{tu-cyan}{RGB}{0,166,214}

\newcommand{\hsp}{\hspace{20pt}}
% \titleformat{\chapter}[hang]{\Huge\headingfont}{\chapternumber\hsp\textcolor{accent2}{|}\hsp}{0pt}{\Huge\headingfont}

% \titleformat{name=\chapter,numberless}[hang]{\Huge\headingfont}{\hsp\textcolor{accent2}{|}\hsp}{0pt}{\Huge\headingfont}

% \titleformat{\section}[block]{\LARGE\headingfont}{\arabic{chapter}.\arabic{section}}{0.4em}{}
% \titleformat{\subsection}[block]{\Large\headingfont}{\arabic{chapter}.\arabic{section}.\arabic{subsection}}{0.4em}{}
% \titleformat{\subsubsection}[block]{\large\headingfont}{\arabic{chapter}.\arabic{section}.\arabic{subsection}.\arabic{subsubsection}}{0.4em}{}
\renewcommand{\arraystretch}{1.05}
\renewcommand{\baselinestretch}{1.1}
\titlespacing{\section}{0pt}{2ex}{1ex}
\titlespacing{\subsection}{0pt}{1ex}{0ex}
\titlespacing{\subsubsection}{0pt}{0.5ex}{0ex}

\renewcommand\cfttoctitlefont{\headingfont\Huge}
\renewcommand\cftloftitlefont{\headingfont\Huge}
\renewcommand\cftlottitlefont{\headingfont\Huge}
\setcounter{lofdepth}{2}
\setcounter{lotdepth}{2}


\setlength{\parindent}{0pt}
\setlength{\parskip}{0.5em}

\captionsetup{width=0.9\textwidth}

%SIuntix settings:
%default: 0V to 10V
%custom: 0 - 10V
\sisetup{range-phrase=--}
\sisetup{range-units=single}
\DeclareSIUnit\years{years}

%For code listings
\definecolor{black}{rgb}{0,0,0}
\definecolor{browntags}{rgb}{0.65,0.1,0.1}
\definecolor{bluestrings}{rgb}{0,0,1}
\definecolor{graycomments}{rgb}{0.4,0.4,0.4}
\definecolor{redkeywords}{rgb}{1,0,0}
\definecolor{bluekeywords}{rgb}{0.13,0.13,0.8}
\definecolor{greencomments}{rgb}{0,0.5,0}
\definecolor{redstrings}{rgb}{0.9,0,0}
\definecolor{purpleidentifiers}{rgb}{0.01,0,0.01}


\lstdefinestyle{csharp}{
language=[Sharp]C,
showspaces=false,
showtabs=false,
breaklines=true,
showstringspaces=false,
breakatwhitespace=true,
escapeinside={(*@}{@*)},
columns=fullflexible,
commentstyle=\color{greencomments},
keywordstyle=\color{bluekeywords}\bfseries,
stringstyle=\color{redstrings},
identifierstyle=\color{purpleidentifiers},
basicstyle=\ttfamily\small}

\lstdefinestyle{c}{
language=C,
showspaces=false,
showtabs=false,
breaklines=true,
showstringspaces=false,
breakatwhitespace=true,
escapeinside={(*@}{@*)},
columns=fullflexible,
commentstyle=\color{greencomments},
keywordstyle=\color{bluekeywords}\bfseries,
stringstyle=\color{redstrings},
identifierstyle=\color{purpleidentifiers},
}

\lstdefinestyle{matlab}{
language=Matlab,
showspaces=false,
showtabs=false,
breaklines=true,
showstringspaces=false,
breakatwhitespace=true,
escapeinside={(*@}{@*)},
columns=fullflexible,
commentstyle=\color{greencomments},
keywordstyle=\color{bluekeywords}\bfseries,
stringstyle=\color{redstrings},
identifierstyle=\color{purpleidentifiers}
}

\lstdefinestyle{vhdl}{
language=VHDL,
showspaces=false,
showtabs=false,
breaklines=true,
showstringspaces=false,
breakatwhitespace=true,
escapeinside={(*@}{@*)},
columns=fullflexible,
commentstyle=\color{greencomments},
keywordstyle=\color{bluekeywords}\bfseries,
stringstyle=\color{redstrings},
identifierstyle=\color{purpleidentifiers}
}

\lstdefinestyle{xaml}{
language=XML,
showspaces=false,
showtabs=false,
breaklines=true,
showstringspaces=false,
breakatwhitespace=true,
escapeinside={(*@}{@*)},
columns=fullflexible,
commentstyle=\color{greencomments},
keywordstyle=\color{redkeywords},
stringstyle=\color{bluestrings},
tagstyle=\color{browntags},
morestring=[b]",
  morecomment=[s]{<?}{?>},
  morekeywords={xmlns,version,typex:AsyncRecords,x:Arguments,x:Boolean,x:Byte,x:Char,x:Class,x:ClassAttributes,x:ClassModifier,x:Code,x:ConnectionId,x:Decimal,x:Double,x:FactoryMethod,x:FieldModifier,x:Int16,x:Int32,x:Int64,x:Key,x:Members,x:Name,x:Object,x:Property,x:Shared,x:Single,x:String,x:Subclass,x:SynchronousMode,x:TimeSpan,x:TypeArguments,x:Uid,x:Uri,x:XData,Grid.Column,Grid.ColumnSpan,Click,ClipToBounds,Content,DropDownOpened,FontSize,Foreground,Header,Height,HorizontalAlignment,HorizontalContentAlignment,IsCancel,IsDefault,IsEnabled,IsSelected,Margin,MinHeight,MinWidth,Padding,SnapsToDevicePixels,Target,TextWrapping,Title,VerticalAlignment,VerticalContentAlignment,Width,WindowStartupLocation,Binding,Mode,OneWay,xmlns:x}
}

\lstdefinestyle{python}{
language=Python,
showspaces=false,
showtabs=false,
breaklines=true,
showstringspaces=false,
breakatwhitespace=true,
escapeinside={(*@}{@*)},
columns=fullflexible,
commentstyle=\color{greencomments},
keywordstyle=\color{bluekeywords}\bfseries,
stringstyle=\color{redstrings},
identifierstyle=\color{purpleidentifiers},
}

%defaults
\lstset{
basicstyle=\ttfamily\scriptsize ,
extendedchars=false,
numbers=left,
numberstyle=\ttfamily\tiny,
stepnumber=1,
tabsize=4,
numbersep=5pt
}
\addbibresource{../../.library/bibliography.bib}
\begin{document}
\section{Introduction}
The goal of the Embedded Real Time Systems course is to get familiar with skills and tasks that are often involved in designing an embedded system.
In order to achieve this, the project goal is to obtain a piece of firmware that allows a quad-rotor drone (the \emph{Quadrupel}) to be controlled in terms of its attitude, rotation and possibly height.
The platform on which this goal is to be achieved is a relatively cheap drone with a custom flight controller board containing several chips, such as a processor and a multitude of sensors.
To provide the drone with input, e.g. the desired attitude set-points, mode of operation, etcetera, a PC (the \emph{ground station} is to be connected with it at all times.
The ground station should run a user interface that provides some visual feedback about the status of the drone, and be able to receive input from at least a joystick and keyboard to provide the drone with commands over the so-called \emph{PC link}.
This PC link will consist of a USB cable at first, providing a direct serial link to the drone, but might later be replaced by the drone's built-in Bluetooth capabilities for wireless flight.

\subsection{Specifications}
As noted, the drone is an affordable, custom built platform, which uses a custom flight controller board consisting of three separate circuit boards, each of which houses part of the required hardware.
A summary of the hardware components is listed below for completeness.

\begin{description}
	\item[Frame] Turnigy Talon V2.0 (550mm)
	\item[Motors] 4x Sunnysky X2212-13 980kV
	\item[ESC] Flycolor 20A BCHeli 204S Opto
	\item[Sensor module] GY-86 (10 DOF)
	\begin{description}
	  	\item[Three-axis gyroscope + triaxial accelerometer] MPU6050
	  	\item[Compass] HMC5883L
	  	\item[Barometer] MS5611
	\end{description}
	\item[RF SoC] nRF51822
	\begin{description}
		\item[Wireless] Bluetooth Low Energy (BLE)
		\item[Microcontroller] ARM Cortex M0 CPU with \SI{128}{kB} flash + \SI{16}{kB} RAM
	\end{description}
\end{description}

\subsection{Requirements}
As has been stated, the controller shall be able to control 3D angular body attitude ($\phi$, $\theta$, $\psi$) and rotation ($p$, $q$, $r$), according to a set-point defined by the joystick connected to the PC and its user interface.
A more extensive specification may be found in the project assignment \cite[3-6]{langendoen2017in4073}, but the most important requirements are listed below for the sake of completeness.

\subsubsection{General requirements}

\begin{description}
	\item[Safety] The drone shall have a \emph{safe state}, which is the default mode in which the drone should start as well as end in both normal operation and in case of a failure. The safe state should preclude the drone from being actuated at all times.
	\item[Emergency stop] The drone shall have a \emph{panic state}, which it should reach whenever the system is aborted. During this state, motor RPM should be reduced to a level at which the motors are unlikely to cause any damage, but that still allows for a relatively soft touchdown, preventing damage to the drone itself.
	\item[Battery health] The drone's battery voltage shall be monitored at all times and should never drop below \SI{10.5}{\volt}. When this level is reached, the drone shall enter panic mode.
	\item[Robust to noise] Sensor data shall be processed in such a way that they allow robust control of the drone. At first the drone's digital motion processor (DMP), will take care of this, but in the final stage of the project this shall be programmed on the microcontroller itself.
	\item[Reliable communication] The PC link shall provide a reliable means of communicating with the drone at all times. The communication protocol should be designed to ensure this is the case.
	\item[Dependability] The embedded system itself shall be reliable as well to ensure safe operation at all times. Whenever erratic behaviour might occur in either the communication link or the drone itself, the system shall detect this and enter panic mode.
	\item[Robustness] Special care must be taken to make the system robust with respect to any possible edge case present in the system that may cause erratic behaviour, e.g. loss of precision due to integer arithmetic, overflows etcetera.
\end{description}

\subsubsection{Functional requirements}

\begin{description}
	\item[PC] The PC shall upload the program to the drone, start it and abort it. Furthermore it shall read input from keyboard and joystick and use this input to send commands to the drone while it is active.
	\item[Drone] The drone shall be able to operate in at least six modes:
	\begin{description}
		\item[Safe mode] The default mode or mode 0, in which the drone shall be idle at all times.
		\item[Panic mode] Mode 1, in which motor RPM is ramped down for a soft touchdown, after which the drone should become idle.
		\item[Manual mode] Mode 2, in which \texttt{lift}, \texttt{roll}, \texttt{pitch} and \texttt{yaw} are actuated directly, without taking into account any sensor feedback.
		\item[Calibration mode] Mode 3, in which sensor offset is measured when the drone is idle, so that it may be used to zero all sensor values.
		\item[Yaw-controlled mode] Mode 4, in which the drone's yaw rate is controlled at the set-point received from the PC.
		\item[Full-controlled mode] Mode 5, in which the drone's roll and pitch angle are controlled at the set-point received from the PC, in addition to its yaw rate.
	\end{description}
	The following modes are optional:
	\begin{description}
		\item[Raw mode] In raw mode the drone shall use unfiltered (untouched by the DMP) accelerometer and gyro data. Readings from both sensors are smoothed and fused by a Kalman filter for roll and pitch, and by a Butterworth filter for yaw.
		\item[Height-control] Whenever toggled, the drone shall use the on-board barometer to maintain its current height.
		\item[Wireless] Whenever toggled, the drone shall communicate with the ground station via Bluetooth in stead of a USB cable.
	\end{description}
	\item[Profiling] Accurate and robust control demand a certain amount of steps per second to function properly. Code should therefore be profiled to determine if this is feasible.
	\item[Logging] The system shall include facilities to write relevant data to the on-board flash chip, which can be downloaded at the end of flight, so that its contents may be analysed after the fact.
\end{description}

\end{document}