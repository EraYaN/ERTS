%!TEX program = xelatex
%!TEX spellcheck = en_GB
\documentclass[final]{article}
\input{../../.library/preamble.tex}
\input{../../.library/style.tex}
\addbibresource{../../.library/bibliography.bib}
\begin{document}
\section{Results}
% Experimental results (list capabilities of your demonstrator)
%Experimental results (list capabilities of your demonstrator)

%the size of the C code

%he control speed of the system (control frequency, the latencies of the various blocks within the control loop)

%TODO[E]: Include the latencies here?
%TODO[E]: Is this also the place for the sensor noise?

The system runs at a control speed of \SI{100}{\hertz}.
The final system has all required modes plus height control (\cref{sec:requirements}).
And additionally phone accelerometer and gyro based input as the RC control signals.\\
The LEDs are programmed as follows.
The blue LED blinks with a frequency of \SI{1}{\hertz} as per the requirements , unless a hard fault occurs.
The red LED blinks when the quad is in PANIC mode, it will also light up if the RC timeout has passed for more than half of it's value, it will light up when a hard fault occurs.
The green LED lights up when the transmit buffer if filled to more than one-fourth of it's capacity.
The yellow LED does the same for the receive buffer.
Both the green and yellow LED help give status in the case of a hard fault.
When yellow lights up solid the sending of debug info over serial is done.
Green will keep blinking.

The final code size is roughly \SI{50}{\kilo\byte}.



\subsection{Module Latency}
\label{ssec:module-latency}
The latencies of all the functions was measured over tens of thousands of data points using the TIMER1 peripheral.
The resulting times in \si{\micro\second} are shown in \cref{tab:function-module-latency}.
As seen the control methods do not add a lot of time to the loop time.
The main time sink is in retrieving the DMP sensor values.
The sampling of the barometer is spread across three loop iterations and is in the base time.

\begin{table}[H]
    \caption{Function/Module latency}
    \label{tab:function-module-latency}
    \centering
    \begin{tabular}{lS[table-format=4.1]s}
    \toprule
    Function/Module                   & {Latency} & \\
    \midrule
    Base & 196.299 & \micro\second \\
    Manual & 8.767 & \micro\second \\
    Yaw & 1.262 & \micro\second \\
    Full & 4.011 & \micro\second \\
    Height & 0.844 & \micro\second \\
    MPU & 3346.344 & \micro\second \\
    RX & 125.880 & \micro\second \\
    Control Logging & 502.256 & \micro\second \\
    Telemetry Logging & 704.864 & \micro\second \\
    Sensor Logging & 640.184 & \micro\second \\
    \bottomrule
    \end{tabular}
\end{table}

\subsection{Sensor Noise}
\label{ssec:sensor-noise}
The sensor values were measured over roughly \SI{20}{\second}.
Then the flash was dumped to the PC and the values analysed to get an idea of the constant noise on the sensors.
The measurement was done with the embedded system newly programmed, after the warm-up period and in a fixed position.
The resulting values are shown in \cref{tab:sensor-noise}.
\begin{table}[H]
    \caption{Sensor noise, the measurement average was used to get the small signal noise.}
    \label{tab:sensor-noise}
    \centering
    \begin{tabular}{lS[table-format=4.2]S[table-format=2.2]S[table-format=1.2]}
    \toprule
    Sensor & {Measurement Avg.} & {Std Dev} & {Peak Noise} \\
    \midrule
    Roll (DMP)      & 1.5880829015544042 & 2.4418125760057356 & 5.5880829015544045 \\
    Pitch (DMP)      & 2.3031088082901556 & 2.1459301577746577 & 4.303108808290156 \\
    Yaw (DMP)      & -0.45595854922279794 & 0.5184455026465902 & 1.544041450777202 \\
    Pressure      & 1003.1891191709844 & 1.4153070489448907 & 4.810880829015559 \\
    Gyro P      & -0.686299948105864 & 0.4651131174832689 & 1.313700051894136 \\
    Gyro Q      & -0.8622210690192008 & 0.6875419292114237 & 2.137778930980799 \\
    Gyro R      & -0.43072132848988065 & 0.5131894516265076 & 1.5692786715101192 \\
    Accelerometer X      & -3314.4566683964713 & 12.573027608906932 & 53.54333160352871 \\
    Accelerometer Y      & 1327.081992734821 & 12.339641781184723 & 46.918007265179085 \\
    Accelerometer Z      & 14902.612869745719 & 20.946644364553567 & 76.61286974571885 \\
    \bottomrule
    \end{tabular}
\end{table}


\end{document}