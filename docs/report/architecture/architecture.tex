%!TEX program = xelatex
%!TEX spellcheck = en_GB
\documentclass[final]{article}
\input{../../.library/preamble.tex}
\input{../../.library/style.tex}
\addbibresource{../../.library/bibliography.bib}
\begin{document}
\section{Architecture}
% Architecture (all software components + interfaces)
The top level architecture shown in \autoref{fig:top-level} gives an overview of how the different parts of the system are connected to each other. In the following sections, each part is explained in more detail.

\begin{figure}[H]
\centering
    \subimport{resources/}{top-level.tikz}
    \caption{Top level overview of the system}
    \label{fig:top-level}
\end{figure}

\subsection{PC-interface}

\subsection{Embedded Software}
In \autoref{fig:quad} a block diagram with the different functions is shown.
The main loop of the embedded software just continuously executes \texttt{busywork()}, where it receives messages if any UART data has arrived and reads out the slow DMP values.
Every time it also checks if the timer flag is set, which indicates that it is time to execute the \texttt{tick()} function.
The internal flow diagram of this function is shown in \autoref{fig:quad-flow}.
In short this function makes sure that the desired behaviour of the current mode is preformed.
So based on the different modes it can for example send heartbeat messages, control the motors or send the data stored in the flash memory over UART to the PC.

\begin{figure}[H]
\centering
    \subimport{resources/}{quad.tikz}
    \caption{Diagram showing different parts of the embedded software on the Quadruple}
    \label{fig:quad}
\end{figure}

\begin{figure}[H]
\centering
    \subimport{resources/}{quad-flow.tikz}
    \caption{Diagram showing the execution path of the \texttt{tick()} function}
    \label{fig:quad-flow}
\end{figure}


\subsection{Communication Library}

\begin{figure}[H]
\centering
    \subimport{resources/}{protocol_message_format.tikz}
    \caption{Generic packet structure. The block is \SI{20}{\byte} wide.}
    \label{fig:packet_format}
\end{figure}
\begin{figure}[H]
\centering
    \subimport{resources/}{protocol_message_examples.tikz}
    \caption{Packet format for 4 of the 11 different \texttt{messageType\_t}'s. The blocks are one \SI{8}{\bit} byte wide and \SI{20}{\byte} high.}
    \label{fig:packet_example}
\end{figure}

\end{document}
