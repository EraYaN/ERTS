%!TEX program = xelatex
%!TEX spellcheck = en_GB
\documentclass[final]{article}
\input{../../.library/preamble.tex}
\input{../../.library/style.tex}
\addbibresource{../../.library/bibliography.bib}
\begin{document}
\section{Architecture}
%TODO[E] Some talk about the interfaces.
% Architecture (all software components + interfaces)
The top level architecture shown in \autoref{fig:top-level} gives an overview of how the different parts of the system are connected to each other. In the following sections, each part is explained in more detail.

\begin{figure}[H]
\centering
    \subimport{resources/}{top-level.tikz}
    \caption{Top level overview of the system}
    \label{fig:top-level}
\end{figure}

\subsection{Dashboard}
\label{ssec:architecture-dashboard}
The ground station software or \emph{dashboard} is a GUI program that consists of a multitude of threads that each perform a different task.
The main tasks of the dashboard program are processing input from keyboard, the GUI, the joystick and even a smartphone; handling communication between the PC and the drone; and visualising any heartbeat data coming from the drone so that the user is informed of its status at all times.
A rough subdivision of the threads and the tasks they execute is shown below.

\begin{itemize}
    \item UI/Main thread
    \begin{itemize}
        \item Configuration management
        \item Global Data reference management
        \item Controller
    \end{itemize}
    \item Input manager thread
    \begin{itemize}
        \item Polling of all acquired joystick and keyboard devices
        \item Generalized Input Event generation
        \item PatchBox to match generalised input event to application action/axis
    \end{itemize}
    \item Serial RX thread
    \item Serial TX thread
    \item Serial Event thread
    \begin{itemize}
        \item Generating generalized serial events
        \item Handling and processing incoming packets
        \item Serializing and sending outgoing packets
    \end{itemize}
    \item RC Timer thread
    \begin{itemize}
        \item Generating periodic remote control value sending events
    \end{itemize}
    \item HTTP server thread
    \begin{itemize}
        \item Handling client connection for phone control
        \item Handling incoming API requests with new control information from a phone
    \end{itemize}
    \item HTTP time-out timer event thread
    \begin{itemize}
        \item Monitor client request intervals to disassociate the client and return control to the joystick.
    \end{itemize}
    \item Framework and UI binding helper threads (many)
        \begin{itemize}
        \item Process UI bindings and animation
        \item Helper thread for internal framework processes
        \item Debug threads
    \end{itemize}
\end{itemize}

\subsection{Embedded Software}
In \autoref{fig:quad} a block diagram with the different functions is shown.
The main loop of the embedded software continuously executes \texttt{busywork()}, which receives messages if any UART data has arrived and reads out the slow DMP values.
It also checks if the timer flag is set, which indicates that it is time to execute the \texttt{tick()} function.
The internal flow diagram of this function is shown in \autoref{fig:quad-flow}.
In short this function makes sure that the desired behaviour of the current mode is performed.
Based on the current mode, it may for example send heartbeat messages, control the motors or send the data stored in the flash memory over UART to the PC.
The entire system acts like a finite state machine (FSM), although it strictly is not.
Any mode transition is performed using the \texttt{set\_mode()} function which makes sure that every transition is pre-defined, making sure that no invalid states occur.

\begin{figure}[H]
\centering
    \subimport{resources/}{quad.tikz}
    \caption{Diagram showing different parts of the embedded software on the Quadrupel}
    \label{fig:quad}
\end{figure}

\begin{figure}[H]
\centering
    \subimport{resources/}{quad-flow.tikz}
    \caption{Diagram showing the execution path of the \texttt{tick()} function}
    \label{fig:quad-flow}
\end{figure}

\newpage
\subsection{Communication Library}
The communication protocol is based on \SI{20}{\byte} fixed-length packets.
This size was chosen to properly fit into \SI{24}{\byte} Bluetooth Low Energy frames.
The packet layout is as shown in \cref{fig:packet-format}
Some examples are shown in \cref{fig:packet-examples}.
\begin{figure}[H]
\centering
    \subimport{resources/}{protocol_message_format.tikz}
    \caption{Generic packet structure. The block is \SI{20}{\byte} wide.}
    \label{fig:packet-format}
\end{figure}
\begin{figure}[H]
\centering
    \subimport{resources/}{protocol_message_examples.tikz}
    \caption{Packet format for 4 of the 11 different \texttt{messageType\_t}'s. The blocks are one \SI{8}{\bit} byte wide and \SI{20}{\byte} high.}
    \label{fig:packet-examples}
\end{figure}

\end{document}
